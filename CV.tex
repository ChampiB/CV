\documentclass[11pt,a4paper,sans]{moderncv}

\moderncvstyle{casual}
\moderncvcolor{burgundy}

\usepackage[utf8]{inputenc}

\usepackage[scale=0.75]{geometry}

\newgeometry{left=0.5in,right=0.5in,top=0.3in,bottom=0.75in}

\name{Théophile}{Champion}
\title{Active Inference - PhD student}
\address{Woolf College, Giles Lane}{Canterbury CT2 7BQ}{UK}
\phone[mobile]{+44(0) 75.52.54.70.07}
\email{tmac3@kent.ac.uk}
\pagenumbering{gobble}

\begin{document}
\makecvtitle
\vspace{-1.3cm}

\section{Education}
\cventry{2019--2022}{Active Inference - PhD}{Kent University}{Canterbury}{}{Active Inference generalises Reinforcement Learning. It uses Variational Inference on a generative model allowing actions to be modelled in a way similar to Planning as Inference. It is generally implemented using Variational Message Passing on Forney Factor Graph.}
\cventry{2018--2019}{Advanced Computer Science - MSc}{Kent University}{Canterbury}{}{Computational Intelligence speciality, two prizes for academic excellence and 92/100 in my examinations. The classes presented an excellent overview of Machine Learning techniques, such as Decision Tree, Ensemble Methods and various other optimisation techniques.}
\cventry{2015--2017}{Bachelor in Information Technologies}{Epitech}{Nantes}{}{Project-based pedagogy aiming to create autonomous and skilled IT professionals. I am familiar with linux and git as well as procedural, object oriented, functional and logic programming.}

\vspace{-0.3cm}
\section{Professional experience}
\cventry{April 2018}{Intern as Data Scientist}{OwnPage}{Paris}{}{Five months during which I have improved a recommender system running on AWS using Spark in Scala and Jupyter Notebook in python.}
\cventry{Sept 2016}{Intern as Web Developer}{Inéance}{Brest}{}{Four months of internship during which I have developed a web application for veterinarians using PHP with ZendFramework 2, HTML, CSS, JavaScript, Ajax,  and MySQL.}

\vspace{-0.3cm}
\section{Computer skills}
\cventry{C language}{Procedural Programming}{Concurrency, parallelism and networking}{}{}{Obtained by developing clients and servers implanting the FTP and IRC protocol, a multi-threaded and GPU-based Deep Learning package and a small virtual machine.}
\cventry{Python Java C++}{Object oriented programming}{Design patterns}{}{}{Acquired by creating my own implementation of Tensorflow and by coding an artificial intelligence to play five-in-a-row using Monte Carlo Tree Search and heuristics on different patterns.}
\cventry{Scala Spark S3 MySQL}{Functional programming and data storage}{Distributed system}{}{}{Learned by developing a veterinarians appointment booking website using ZendFramework2 and a recommender system for newsletters' articles using Singular Value Decomposition, Spark in Scala and EC2 instances.}

\vspace{-0.3cm}
\section{Machine Learning skills}
\cventry{Weka}{General Machine Learning}{Tree Based Models, Evolutionary Algorithms and Clustering}{}{}{Obtained by watching Andrew Ng's Online MOOC on Machine Learning, playing with Weka to understand overfitting and underfitting in Decision Trees, reading Christopher Bishop's book on Pattern Recognition and Machine Learning.}
\cventry{TensorFlow}{Deep Learning}{Image processing and time series}{}{}{I have been reading about basic layers such as Dense, Convolutional, Up-Conv, Recurrent, GRU and LSTM. More complex architectures for image classification (e.g. ResNet, VGG, AlexNet and GoogleNet), image segmentation (e.g. UNet and LinkNet), language translation (e.g. encoded/decoder architecture) and popular techniques (e.g. Variational Autoencoders and Generative Adversarial Networks). I have been using Tensorflow for Kaggle challenges.}
\cventry{My own toolbox}{Probabilistic Modelling}{Exact Inference, Variational Inference and Markov Chain Monte Carlo}{}{}{During my PhD, I have been studying exact inference methods (e.g. sum-product algorithm), approximate inference methods (e.g. Expectation Maximisation, Active Inference and Variational Message Passing) as well as sampling based methods (e.g. Markov Chain Monte Carlo).}

\vspace{-0.3cm}
\section{Languages}
\cvitemwithcomment{French}{C2}{Native speaker}{}{}
\cvitemwithcomment{English}{C2}{Fluent}

\end{document}
