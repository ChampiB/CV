\documentclass[11pt,a4paper,sans]{moderncv}

\moderncvstyle{casual}
\moderncvcolor{burgundy}

\usepackage[utf8]{inputenc}

\usepackage[scale=0.75]{geometry}

\newgeometry{left=0.5in,right=0.5in,top=0.3in,bottom=0.75in}

\name{Théophile}{Champion}
\title{PhD candidate in Active Inference (Reinforcement Learning and Generative AI)}
\address{28 Exeter Road}{Birmingham B29 6EU}{UK}
\phone[mobile]{+33(0)6.17.43.48.60}
\email{txc314@student.bham.ac.uk}
\pagenumbering{gobble}

\begin{document}
\makecvtitle
\vspace{-1.3cm}

\section{Education}
\cventry{2024--2025}{Machine Learning - PhD}{University of Birmingham}{Birmingham}{}{Applying machine learning methods such as Monte Carlo Tree Search and Variational Auto-Encoders to scale up Active Inference. I published six first author papers, and submitted two others, for a total of 71 citations.}
\cventry{2019--2023}{Machine Learning - PhD}{University of Kent (transferred to University of Birmingham)}{Canterbury}{}{}
\cventry{2018--2019}{Advanced Computer Science - MSc}{University of Kent}{Canterbury}{}{Computational Intelligence speciality, two prizes for academic excellence and 92/100 in my examinations. The classes presented an excellent overview of Machine Learning techniques, such as Decision Tree, Ensemble Methods and various other optimisation techniques.}
\cventry{2015--2017}{Bachelor in Information Technologies}{Epitech}{Nantes}{}{Project-based pedagogy aiming to create autonomous and skilled IT professionals. I am familiar with linux and git as well as procedural, object oriented, functional and logic programming.}

\vspace{-0.3cm}
\section{Publications}
\cventry{2024}{\small{Structure learning with Temporal Gaussian Mixture for model-based Reinforcement Learning.}}{}{}{}{\footnotesize{Th\'eophile Champion, Howard Bowman, and Marek Grze\'s, Submitted.}}
\cventry{2023}{\small{Reframing the Expected Free Energy: Four Formulations and a Unification.}}{}{}{}{\footnotesize{Th\'eophile Champion, Marek Grze\'s, and Howard Bowman, Submitted. (5 citations)}}
\cventry{2023}{\small{Deconstructing deep active inference.}}{}{}{}{\footnotesize{Th\'eophile Champion, Marek Grze\'s, Lisa Bonheme, and Howard Bowman, Neural Computation. (4 citations)}}
\cventry{2022}{\small{Branching time active inference: Empirical study and complexity class analysis.}}{}{}{}{\footnotesize{Th\'eophile Champion, Howard Bowman, and Marek Grze\'s, Neural Networks. (17 citations)}}
\cventry{2022}{\small{Branching time active inference: The theory and its generality.}}{}{}{}{\footnotesize{Th\'eophile Champion, Lancelot Da Costa, Howard Bowman, and Marek Grze\'s, Neural Networks. (20 citations)}}
\cventry{2022}{\small{Branching Time Active Inference with Bayesian Filtering.}}{}{}{}{\footnotesize{Th\'eophile Champion, Marek Grze\'s, and Howard Bowman, Neural Computation. (5 citations)}}
\cventry{2022}{\small{Multi-modal and multi-factor branching time active inference.}}{}{}{}{\footnotesize{Th\'eophile Champion, Marek Grze\'s, and Howard Bowman, Neural Computation. (1 citation)}}
\cventry{2021}{\small{Realizing active inference in variational message passing: The outcome-blind certainty seeker.}}{}{}{}{\footnotesize{Th\'eophile Champion, Marek Grze\'s, and Howard Bowman, Neural Computation. (19 citations)}}

\vspace{-0.3cm}
\section{Professional experience}
\cventry{January 2023}{Research Assistant}{United Kingdom Ministry of Defence - DSTL}{Canterbury}{}{One year during which I have trained reinforcement learning models to defend computer networks using Hydra and PyTorch. The benchmarking and analysis of the network simulators was based on Pandas and Matplotlib.}
\cventry{June 2022}{IT consulting}{Digital Gaia}{Denver}{}{Eleven months mission aiming to develop a framework for federated probabilistic inference based on NumPyro in python, and collaborating with domain experts to create models of agroforestry ecosystems.}
\cventry{April 2018}{Intern as Data Scientist}{OwnPage}{Paris}{}{Five months during which I have improved a recommender system running on AWS using Spark in Scala and Jupyter Notebook in python.}
\cventry{Sept 2016}{Intern as Web Developer}{Inéance}{Brest}{}{Four months of internship during which I have developed a web application for veterinarians using PHP with ZendFramework 2, HTML, CSS, JavaScript, Ajax,  and MySQL.}

\vspace{-0.3cm}
\section{Languages}
\cvitemwithcomment{French}{Native speaker}{}{}{}
\cvitemwithcomment{English}{Fluent}{}

\vspace{-0.3cm}
\section{Machine Learning skills}
\cventry{PyTorch TensorFlow}{Deep Learning}{Image processing and time series}{}{}{I have been reading about basic layers such as Dense, Convolutional, Up-Conv, Recurrent, GRU and LSTM. More complex architectures for image classification (e.g. ResNet, VGG, AlexNet and GoogleNet), image segmentation (e.g. UNet and LinkNet), language translation (e.g. encoded/decoder architecture) and popular techniques (e.g. Variational Autoencoders and Generative Adversarial Networks).}
\cventry{NumPyro}{Probabilistic Modelling}{Exact Inference, Variational Inference and Markov Chain Monte Carlo}{}{}{During my PhD, I have been studying exact inference methods (e.g. sum-product algorithm), approximate inference methods (e.g. Expectation Maximisation, Active Inference and Variational Message Passing) as well as sampling based methods (e.g. Markov Chain Monte Carlo).}
\cventry{Weka}{General Machine Learning}{Tree Based Models, Evolutionary Algorithms and Clustering}{}{}{Obtained by watching Andrew Ng's Online MOOC on Machine Learning, playing with Weka to understand overfitting and underfitting in Decision Trees, reading Christopher Bishop's book on Pattern Recognition and Machine Learning.}

\vspace{-0.3cm}
\section{Computer skills}
\cventry{Python Java C++}{Object oriented programming}{Design patterns}{}{}{Acquired by creating my own implementation of Tensorflow and by coding an artificial intelligence to play five-in-a-row using Monte Carlo Tree Search and heuristics on different patterns.}
\cventry{C language}{Procedural Programming}{Concurrency, parallelism and networking}{}{}{Obtained by developing clients and servers implementing the FTP and IRC protocol, a multi-threaded and GPU-based Deep Learning package and a small virtual machine.}
\cventry{Scala Spark S3 MySQL}{Functional programming and data storage}{Distributed system}{}{}{Learned by developing a veterinarians appointment booking website using ZendFramework2 and a recommender system for newsletters' articles using Singular Value Decomposition, Spark in Scala and EC2 instances.}

\vspace{-0.3cm}
\section{Mathematical skills}
\cventry{Expert}{Probability and Calculus}{}{}{}{Mastered through online courses, reading books, deriving results stated in scientific publications, implementing custom algorithms grounded in Bayesian probability and custom differentiable operators for automatic differentiation frameworks.}
\cventry{Proficient}{Linear algebra, Information theory, Statistics, and Graph theory}{}{}{}{Learned by following online courses, teaching other researchers in reading groups, and  reading scientific papers.}
\cventry{Novice}{Fields studied out of curiosity}{}{}{}{Measure theory, Abstract algebra, Geometry, Algebraic geometry, Differential geometry, Topology, Group theory, Game theory, Set theory, and Complex analysis.}

\end{document}
